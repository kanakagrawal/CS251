\section{Inorder Binary Search Traversal}
The basic inorder binary search traversal follows the following implementation:
\begin{itemize}
	\item Traverse the left subtree by recursively calling the in-order function.
	\item Display the data part of the root (or current node).
	\item Traverse the right subtree by recursively calling the in-order function.
\end{itemize}
There are many ways to implement it such as by recursion, stack, etc.
The recursive implementation is very trivial. Following is the method to implement using stacks without recursion:
\begin{itemize}
	\item Create an empty stack S.
	\item Initialize current node as root.
	\item Push the current node to S and set current = current-$>$left until current is NULL.
	\item If current is NULL and stack is not empty then \\
     a) Pop the top item from stack.\\
     b) Print the popped item, set current = popped item-$>$right \\
     c) Go to step 3.
    \item If current is NULL and stack is empty then we are done.	
\end{itemize}
\cite{knuth1973vol}