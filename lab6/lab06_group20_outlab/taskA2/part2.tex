\section{Shortest Path}
\textbf{Dijkstra's algorithm} is an algorithm for finding the shortest paths between nodes in a graph. Textual description of the algorithm is as follows:\\[1em]
Let the node at which we are starting be called the initial node. Let the distance of node Y be the distance from the initial node to Y. Dijkstra's algorithm will assign some initial distance values and will try to improve them step by step.\\[0.5em]
\begin{itemize}
	\vspace{-1em}
	\item Assign to every node a tentative distance value: set it to zero for our initial node and to infinity for all other nodes.
	\item Set the initial node as current. Mark all other nodes unvisited. Create a set of all the unvisited nodes called the unvisited set.
	\item For the current node, consider all of its unvisited neighbors and calculate their tentative distances. Compare the newly calculated tentative distance to the current assigned value and assign the smaller one. For example, if the current node A is marked with a distance of 6, and the edge connecting it with a neighbor B has length 2, then the distance to B (through A) will be 6 + 2 = 8. If B was previously marked with a distance greater than 8 then change it to 8. Otherwise, keep the current value.
	\item When we are done considering all of the neighbors of the current node, mark the current node as visited and remove it from the unvisited set. A visited node will never be checked again.
	\item If the destination node has been marked visited (when planning a route between two specific nodes) or if the smallest tentative distance among the nodes in the unvisited set is infinity (when planning a complete traversal; occurs when there is no connection between the initial node and remaining unvisited nodes), then stop. The algorithm has finished.
	\item Otherwise, select the unvisited node that is marked with the smallest tentative distance, set it as the new \'current node\', and repeat.
\end{itemize}
\cite{dijkstra1959note}